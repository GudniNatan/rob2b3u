\section{Prófanir}
Hér skal gera lýsingu á prófunum á kerfinu . Til dæmis ef þið eruð með Arduino sem vefþjónn sem byrtir gildi frá hitamæli, rakamæli og gas mæli þá gæti prófunin verið svona: 1. prófun á vef, 2. prófun á hitamæli, .prófun á gasmæli hvert og eitt prófað sér áður en allt er sett saman og þá er gerð prófun á öllu kerfinu

Því miður fór eitthvað úrskeiðis í sendingu á pörtum til landsins, og við höfðum ekki tíma til þess að setja verkefnið saman. Við höfðum aftur á móti tíma til þess að setja upp allan hugbúnað sem þurfti fyrir verkefnið. Guðni gerði það að mestu leyti.

Ég setti upp mína eigin raspberry pi með retropie stýrikerfinu fyrst. Það tók tæpa 13 klukkutíma að downloada og installa öllum mismunandi pökkunum. Mest gerðist á sjálfu sér, en eitthvað fór úrskeiðis, og ég þurfti að fara gegnum allt til að finna hvaða pakka það vantaði (það var VLC). Eftir að allt var komið í gang þurfti ég bara að image-a sd kortið, og þá var allt tilbúið hugbúnaðarmegin.

Ég er líka að búa til tölvuleik fyrir annan áfanga í python og forritaði inn í hann stuðning fyrir leikjafjarstýringar og setti hann inn á pi-ið.