\section{Inngangur}
Hér skal gera lýsingu á verkefninu þ.e hvað,  hvernig og  hvaða forritunarmál, fyrir hverja og hvaða notagildi verkefnið hefur. Minnst 500 orð. Notagildi skiptir miklumáli, reynið að sjá fyrir ykkur hverjir geti notað vélmennið ykkar og í hvaða tilgangi.  Þá kemur í ljós að 500 orð er frekar lítið :-) Hér er gott að byrja á því að lesa til um Arduino en allt hjá þeim er open-sourse og svo er hægt að lesa sér til um efnið í útgefnum bókum sem "programming Arduino \cite{monk} Skoðið vel heimildaskrá og skránna mybib.bib. Hér er gott að lýsa högun kerfisins með orðum og mynd sem þið getið gert í draw.io sjá mynd:

\begin{figure}[h]
\includegraphics[scale=.3]{img/system}
\end{figure}
\section{Ótitlað RetroPi verkefni}
Við ætlum að gera ferða-leikjatölvu. Hún verður byggð á raspberry pi 3 model B og mun keyra RetroPie stýrikerfið. Hún mun geta spilað alls konar leiki frá Mario til Doom. Tölvan notar snertiskjá sem verður vonandi hægt verður að nota í leikjum (sem auka takka eða eitthvað), það verður rafhlaða, og hátalarar fyrir hljóð. Hægt verður að hlaða hana með micro-USB snúru. Til þess að gera þetta allt saman þurfum við að leysa ýmis vandamál eins og hvaða búnað það væri best að nota, samsetningu og uppsetningu á hugbúnaði.

\textbf{Hönnun}
Áður en við getum í raun hafið verkefnið verðum við fyrst að hanna vélina. Það er því miður mjög erfitt að breyta til eftir að búið er að festa kaup á pörtum.
