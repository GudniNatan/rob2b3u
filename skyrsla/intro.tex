\section{Inngangur}
Hér skal gera lýsingu á verkefninu þ.e hvað,  hvernig og  hvaða forritunarmál, fyrir hverja og hvaða notagildi verkefnið hefur. Minnst 500 orð. Notagildi skiptir miklumáli, reynið að sjá fyrir ykkur hverjir geti notað vélmennið ykkar og í hvaða tilgangi.  Þá kemur í ljós að 500 orð er frekar lítið :-) Hér er gott að byrja á því að lesa til um Arduino en allt hjá þeim er open-sourse og svo er hægt að lesa sér til um efnið í útgefnum bókum sem "programming Arduino \cite{monk} Skoðið vel heimildaskrá og skránna mybib.bib. Hér er gott að lýsa högun kerfisins með orðum og mynd sem þið getið gert í draw.io sjá mynd:

\begin{figure}[h]
\includegraphics[scale=.3]{img/system}
\end{figure}
\section{Um verkefnið}
Við ætlum að gera ferða-leikjatölvu. Hún verður byggð á raspberry pi 3 model B og mun keyra RetroPie stýrikerfið. Hún mun geta spilað alls konar leiki frá Mario til Doom. Tölvan notar snertiskjá sem verður vonandi hægt verður að nota í leikjum (sem auka takka eða eitthvað), það verður rafhlaða, og hátalarar fyrir hljóð. Hægt verður að hlaða hana með micro-USB snúru. Til þess að gera þetta allt saman þurfum við að leysa ýmis vandamál eins og hvaða búnað það væri best að nota, samsetningu og uppsetningu á hugbúnaði.

\subsection{Hönnun}

Áður en við getum í raun hafið verkefnið verðum við fyrst að hanna vélina. Það er því miður mjög erfitt að breyta til eftir að búið er að festa kaup á pörtum, svo við verðum að vanda okkur við valið.

\subsection{Samsetning}

Eftir að við erum komnir með alla partana og hönnunin er tilbúin er samsetningin næst. Við þurfum örugglega að lóða saman víra og skrúfa saman parta.

\subsection{Hugbúnaður}

Þegar við erum búnir að setja saman alla partana þurfum við að huga að hugbúnaðinum. Við ætlum að nota RetroPi stýrikerfið, sem inniheldur heilann helling af emulator-um, sem spila gamla leiki. Við þurfum að gera einhver script til að passa upp á að allt saman hagi sér rétt, við viljum að þú getir stjórnað öllu saman bara með tökkunum á unitinu, án þess að þurfa að tengja auka lyklaborð við eða eitthvað. Svo þurfum við líka að passa að takkarnir, hátalarar og skjárinn tengjast rétt við stýrikerfið.

\subsection{Af hverju þetta verkefni?}

Útaf því það er skemmtilegt!

Við völdum þetta verkefni af því að við höfum gaman af tölvuleikjum og okkur langaði til að gera leikjatölvu. Okkur fannst þetta spennandi hugmynd.

\textbf{Nothæfi}

Þessa leikjatölvu væri hægt að nota í, eins og nafnið gefur til, ýmsa tölvuleiki. Það væri hægt að fjöldaframleiða og selja svona tölvur. Á markaðinum núna eru til ýmsar leikjatölvur sem spila gamla leiki, en þær spila yfirleitt bara leiki frá einu fyrirtæki og svo þarftu alltaf að tengja það við sjónvarpið. Með okkar tölvu gætir þú spilað nánast hvaða gamla leik sem er (svo lengi sem þú átt ROM-ið) og tekið þá með þér. Þetta er betra en síminn þinn því hann kemur ekki með neinum tökkum sem hægt er að nota til að stjórna leiknum, bara snertiskjá.